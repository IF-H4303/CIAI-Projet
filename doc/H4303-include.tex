\documentclass[a4paper, 9pt]{article}

% Faire des marges un peu moins large que celles par defaut
\usepackage[top=20mm, bottom=15mm, left=10mm, right=10mm]{geometry}
%\usepackage{ucs}
\usepackage[utf8]{inputenc} % Pour l'encodage 
% Reconnaitre les carateres accentues dans le source.
\usepackage[T1]{fontenc} 
% Meilleurs polices
\usepackage[sc]{mathpazo}
%\usepackage{concmath}
\usepackage[francais]{babel}
% Insertion d'images
\usepackage{graphicx}
% Pour le listing de code
\usepackage{listings}
% Pour la coloration syntaxique
\usepackage{xcolor}
% Pour fixer l'interlignage
\usepackage{setspace} 
% Pour faire un index (ici glossaire)
\usepackage{makeidx}
% Pour gérer les liens internes et les URL cliquables
\usepackage{url}
% Pour les headers et footers
\usepackage{fancyhdr}
% Pour le logo en haut a droite
\usepackage{eso-pic} 
% Pour l'enroulement du texte autour des figures
\usepackage{wrapfig}
% Pour la couverture en PDF pleine page
\usepackage{pdfpages} 
% Pour la biblio bibtex
\usepackage{bibunits}
% Pour gérer les éléments flottants
\usepackage{float}
% Pour les cadres à ombrage du glossaire
\usepackage{fancybox}
% Pour faire des sous-figures correctement numérotés
\usepackage{subfigure}
% Pour mettre les liens cliquables
\usepackage{hyperref}
\usepackage{verbatim}
\usepackage{algorithmic}
\usepackage{ifthen}
\usepackage{xargs}
\NoAutoSpaceBeforeFDP

  % Configuration de la coloration syntaxique du code
  \definecolor{colKeys}{rgb}{0,0,1}
  \definecolor{colIdentifier}{rgb}{0,0,0}
  \definecolor{colComments}{rgb}{0.1,0.6,0.3}
  \definecolor{colString}{rgb}{0.6,0.1,0.1}
  
\lstdefinelanguage{JavaScript}{
  keywords={typeof, new, true, false, catch, function, return, null, catch, switch, var, if, in, while, do, else, case, break},
  ndkeywords={class, export, boolean, throw, implements, import, this},
  identifierstyle=\color{black},
  sensitive=false,
  comment=[l]{//},
  morecomment=[s]{/*}{*/},
  morestring=[b]',
  morestring=[b]",
}

\setlength{\headheight}{14.5pt}

% Configuration des options 
\lstset{%
    identifierstyle=\color{colIdentifier},%
    basicstyle=\ttfamily\scriptsize, %
    keywordstyle=\color{colKeys},%
    stringstyle=\color{colString},%
    commentstyle=\color{colComments},%
    columns = flexible,%
    %tabsize = 8,%
    showspaces = false,%
    numberstyle=\tiny,%
    frame = single,frameround=tttt,%
    breaklines = true, breakautoindent = true,%
    captionpos = b,%
    xrightmargin=10mm, xleftmargin = 15mm, framexleftmargin = 7mm,%
    includerangemarker=false,
    rangeprefix=/*\{,
    rangesuffix=\}*/,
  }%

% Permet l'ajout de code par insertion du fichier le contenant
% Coloration + ajout titre
% Les arguments sont :
% $1 : nom du fichier à inclure
% $2 : le type de langage (C++, C, Java ...)
\newcommand{\addCode}[5]{%


  % Configuration des options 
  \lstset{%
    language = #1,
    numbers = #5,%
   }%
    \begin{center}
    \lstinputlisting[linerange=#3-#4]{#2}
    \end{center}
}

%Permet d'ajouter un exemple

\newcommand{\wbalTwo}[2][Wikimedia]{This is the Wikibook about LaTeX supported by {#1} and {#2}!}

\newcommandx{\addExample}[9][5 = , 6 = ,7= ,8= ,9=none ]
{
    \begin{center}
    \footnotesize
        \definecolor{lightlgray}{rgb}{0.95,0.95,0.95}
        \fcolorbox{black}{lightlgray}{
            \begin{minipage}{0.9\linewidth}
                \setlength{\fboxrule}{1pt}
                \framebox{{\Large\texttt{#1}}}
                \setlength{\fboxrule}{0.4pt}
                \footnotesize
                \begin{description}
                    \item[Syntaxe] ~\\
                        \texttt{#2}
                    \item[Description]
                    #3
                    \item[Valeur de retour]~\\
                    #4
                      \ifthenelse{\equal{#5}{}}
                      {}
                      {
                        \item[Exemple d'utilisation] ~\\
                        \begin{minipage}{0.9\linewidth}
                        \addCode{#5}{#6}{#7}{#8}{#9}
                        \end{minipage}
                      }

                  
                \end{description} 
            \end{minipage}
        } %end fcolorbox
    \end{center}
    \normalsize
}

% Couleur des url et des liens internes.
\hypersetup{urlcolor=blue,linkcolor=black,citecolor=black,colorlinks=true}

%headers, footers
\pagestyle{fancy}
\rhead{INSA de Lyon -- 2011/2012}
\cfoot{\thepage}
\lfoot{[Héxanome H4303]}
